% Costos y Precios

\section{¿Cuál es la diferencia entre Costo y Precio?}

\subsection{Costo}

Es todo lo que te sale producir o prestar un servicio. Es lo que invertís para que el producto exista:

\begin{itemize}
\item Materia prima: insumos, ingredientes, repuestos, telas, etc.
\item Mano de obra: tu tiempo o el de otras personas.
\item Herramientas o materiales indirectos: tijeras, cinta, nafta, gas, detergente, electricidad.
\item Traslado o logística: delivery, combustible, pasajes.
\end{itemize}

\subsubsection{Conocerlo es clave}

Conocerlo es clave para:

\begin{itemize}
\item Fijar precios correctamente.
\item Evitar pérdidas.
\item Saber cuánto ganás realmente.
\item Tomar decisiones estratégicas (¿me conviene producir más?, ¿tercerizar?, ¿aumentar el precio?, etc.)
\end{itemize}

\subsection{Precio}

Es lo que le vas a cobrar al cliente. Es el costo más tu ganancia.

Tu precio debe cubrir tus gastos y además generar un margen que justifique tu esfuerzo y te permita crecer.

\begin{calloutnote}
Fórmula básica: $Precio = Costo + Ganancia$
\end{calloutnote}

\section{Tipos de Costos}

\subsection{Costos Fijos}

Son aquellos que no varían aunque produzcas más o menos. Los pagás sí o sí.

Ejemplos:

\begin{itemize}
\item Alquiler de tu local o taller.
\item Sueldo fijo (si tenés personal)
\item Servicios básicos (agua/luz) si son mensuales y constantes.
\item Herramientas o máquinas (amortizados en el tiempo)
\end{itemize}

Ejemplo:

Pagás Gs. 1.000.000 al mes de alquiler. Vendas 10 productos o 100, ese costo no cambia.

\subsection{Costos Variables}

Aumentan o disminuyen según la cantidad que producís o vendés.

Ejemplos:

\begin{itemize}
\item Materia prima (telas, ingredientes, cables, etc.)
\item Energía por uso (gas, electricidad industrial)
\item Empaques
\item Comisiones por venta (si usás plataformas o revendedores)
\end{itemize}

Ejemplo:

Por cada vela artesanal, necesitás una mecha de Gs. 2.000. Si hacés 20 velas, gastás 40.000 Gs.

\begin{calloutnote}
Este costo sí depende de cuántas unidades hacés.
\end{calloutnote}

\subsection{Costos indirectos}

No se asocian directamente a un producto, pero son necesarios para la operación general.

Ejemplos:

\begin{itemize}
\item Limpieza del espacio.
\item Marketing y publicidad.
\item Internet.
\item Transporte/logística de varios pedidos juntos.
\end{itemize}
