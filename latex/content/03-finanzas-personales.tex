% Finanzas Personales

\section{¿Por qué son importantes las finanzas personales?}

Porque el emprendimiento \textbf{nace y crece} en el entorno familiar. Si \textbf{no hay control} de las finanzas personales:

\begin{itemize}
\item Se confunden los \emph{ingresos del negocio} con lo que va saliendo del bolsillo sin registro.
\item No sabés si realmente estás \emph{ganando o perdiendo}.
\item Es \emph{difícil ahorrar, invertir} o mejorar tu calidad de vida.
\end{itemize}

\begin{calloutimportant}
¡Y todo eso desmotiva!
\end{calloutimportant}

\section{Claves para una buena gestión financiera}

\subsection{Separá tus cuentas}

\begin{itemize}
\item Si podés, usá dos billeteras o cuentas separadas (una para el emprendimiento, otra para lo personal)
\item No pagues las compras del súper con el dinero de las ventas sin llevar un control.
\end{itemize}

\begin{callouttip}
Anotá cuánto dinero sacás del negocio por día para gastos personales o del hogar. Vas a ver cuánto se te va sin darte cuenta.
\end{callouttip}

\subsection{Controlá tus ingresos y egresos}

Usá una hoja, una libreta o una app sencilla (como \emph{Monefy} o \emph{Spendee}) para anotar:

\begin{table}[htbp]
  \centering
  \begin{tabular}{llll}
    \toprule
    \textbf{Día} & \textbf{Ingreso} & \textbf{Gasto} & \textbf{Motivo} \\
    \midrule
    & & & \\
    \bottomrule
  \end{tabular}
  \caption{Control de ingresos y egresos}
\end{table}

Este simple hábito te da claridad total sobre tu dinero.

\subsection{Asigná un sueldo mensual}

\begin{itemize}
\item Calculá cuánto necesitás para cubrir tus gastos personales.
\item Luego, fijá un monto fijo mensual que vas a transferirte del negocio (como si fueras tu propio colaborador/a)
\item Si el negocio crece, podemos aumentarlo.
\end{itemize}

Esto ayuda a dejar de a poco y mantener la salud financiera del emprendimiento.

\subsection{Armá un fondo de ahorro o emergencia}

\begin{itemize}
\item Separá aunque sea un 5\% de tus ingresos cada semana.
\item Guardalo para imprevistos (una máquina que se rompe, enfermedad, baja en ventas)
\end{itemize}

\textbf{Ejemplo}: Si ganas 200.000 Gs. por semana, separá 10.000 Gs. en una alcancía o cuenta digital.

\section{¿Qué son los gastos hormiga?}

Son pequeños gastos diarios que hacemos automática o por costumbre, y que en apariencia no afectan mucho... pero cuando los sumás al mes o al año, pueden representar una pérdida importante de dinero.

Se llaman así porque, como las hormigas, trabajan en silencio y en grupo: una no molesta, pero cientas sí.

Se detectan anotando todos los gastos que hacemos de manera diaria.
